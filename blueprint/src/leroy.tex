\chapter{Leroy}


\section{F star}

\begin{definition}[Frame]
    \label{lem:frame}
    (Already in Mathlib )A Frame can be viewed as a Category.
    There exists a morphism between $A$ and $B$ iff $A \le B$.
    \leanok
\end{definition}

\begin{definition}[Top -> Frame]
    \label{def:top_frame_func}
    \uses{lem:frame}
    (Brauchen wir das überhaupt?) There exists a contravariant Functor from a topological Space to the corresponding Frame Category with open Sets as objects. $f: X \rightarrow O(X)$
\end{definition}

\begin{definition}[$f^*$ and $f_*$]
    \label{def:f_star}
    \uses{def:top_frame_func}
    \lean{f_obenstern}
    \leanok
    For every continuous Function $f : X \rightarrow Y$ between topological Spaces, there exists a pair of functors $(f^*,f_*)$.
    \begin{gather*}
        f* = f^{-1} : O(Y) \rightarrow O(X)\\
        f_* : O(X) \rightarrow O(Y) := A \mapsto \bigcup_{f^*(v) \le A} v\\
    \end{gather*}
\end{definition}

\begin{lemma}[$f^*$ commutes]
    \label{lem:f_star_comm}
    \uses{def:f_star}

    $f^*$ commutes with finite meets and arbitrary joins. This is the same as saying that the Frame Category of open Sets has all small coproducts and all finite Limits. \href{https://ncatlab.org/nlab/show/frame}{nlab}
\end{lemma}

\begin{lemma}[Homsets]
    \label{lem:f_star_homsets}
    \uses{lem:f_star_adj}
    \leanok
    (Already in Mathlib)
    \[Hom_{O(Y)}(f*(V), A) = Hom_{O(x)}(v, f_*(A))\]
\end{lemma}

\begin{lemma}[$f^* \dashv f_*$]
    \label{lem:f_star_adj}
    \uses{def:f_star}
    \lean{f_adj}
    \leanok
    $f^*$ is the right adjoint to $f_*$
\end{lemma}
\begin{proof}
    \leanok
\end{proof}

\begin{lemma}[triangle]
(Mclane p. 485)

    The triangular identities reduce to the following equalities:
    \[f^*f_*f^* = f^* \quad \text{and} \quad f_*f^*f_* = f_*\]
    \lean{triangle_one}
    \leanok
    \label{lem:triangle}
\end{lemma}
\begin{proof}
    \uses{lem:f_star_adj}

    This follows from the triangular identities of the adjunction.
    \leanok
\end{proof}


\section{Embedding}

\begin{lemma}[Embedding]
(Leroy Lemme 1)
    \label{lem:embedding}
    \uses{def:f_star,lem:triangle}
    \lean{f_injective_surjective}
    The following arguments are equivalent:
    \begin{enumerate}
        \item $f^*$ is surjective
        \item $f_*$ is injective
        \item $f^{*}f_* = 1_{O(X)})$
    \end{enumerate}
\end{lemma}
\begin{proof}
    This follows from the triangular identities. \leanok
\end{proof}

\begin{definition}[Embedding]
    \label{def:embedding}
    \uses{lem:embedding}
    An embedding is a morphism that satisfies the conditions of \cref{lem:embedding}
    \lean{Leroy_Embedding}
    \leanok
\end{definition}


\section{Subspaces}

\begin{definition}[Nucleus]
    \label{def:nucleus}
    \uses{lem:frame}
    \lean{Nucleus}
    \leanok
    A nucleus is a map $e : O(E) \rightarrow O(E)$ with the following three properties:
    \begin{enumerate}
        \item $e$ is idempotent
        \item $U \le e U$
        \item $e(U \cap V) = e(U) \cap e(V)$
    \end{enumerate}
\end{definition}

\begin{lemma}[Nucleus]
(Leroy Lemme 3)
    \label{lem:nucleus}
    \uses{def:f_star,def:nucleus,def:embedding}
    Let $e : O(E) \rightarrow O(E)$ be a nucleus. Then there is a space $X$ and a morphism $f: X \rightarrow E$ such that $e = f_*f^*$. (The same holds for embeddings ??)
\end{lemma}

\begin{definition}[Subframe]
(Leroy CH 3)
    \label{lem:subframe}
    \uses{lem:nucleus}
    A subspace $Y \subset X$ is defined by a nucleus $e_Y: O(X) \rightarrow O(X)$, such that $O(Y) = Im(e_Y) = \{U \in O(X) | e_Y(U) = U\}$.
    The corresponding embedding (???) is $i_X : O(Y) \rightarrow O(X)$. $i^*_X(V) = e_X(V)$, $(i_X)_*(U) = U$
\end{definition}

\begin{definition}[Subframe Inclusion]
(Stimmt das?)(Leroy Ch 3)
    \label{def:subframe_inclusion}
    \uses{lem:subframe}
    $X \subset Y$ if $e_Y \le e_X$
\end{definition}

\begin{lemma}[factorisation]
(Leroy Lemme 2)
    \label{lem:factorisation}
    \uses{def:f_star, def:embedding}
    Let $i : X \rightarrow E$ be an embedding and $f: Y \rightarrow E$ be a morphism of spaces. To have $f$ factor through $i$, it is necessary and sufficient that $i_*i^*(V) \le f_*f^*(V) for all V \in O(E)$.
\end{lemma}


\begin{lemma}[Familiy of subspaces]
(Leroy CH 4)
    \label{lem:family_of_subspaces}
    \uses{lem:nucleus}
    Let $X_i$ be a family of subspaces of $E$ and $e_i$ be the corresponding nuclei. For every $V \in O(E)$, let $e(V)$ be the onion of all $W \in O(E)$ which are contained in every $e_i(V)$. Then \\
    \begin{enumerate}
        \item $e$ is the corresponding nucleus of a subspace $X$ of $E$
        \item a subspace $Z$ of $E$ contains $x$ if and only if it contains all $X_i$ x is thus called the union of $X_i$ denoted by $\bigcup_i X_i$
    \end{enumerate}
\end{lemma}

\subsection{Direct Images}
\begin{definition}[Direct Images]
    \uses{lem:subframe}
    \label{def:direct_images}
    Let $f : E \rightarrow F$ be a morphism of spaces. The map $f_*f^* : O(F) \to O(F)$ is the nucleus of the subspace $Im(f)$ of $F$. By (lemma 2), $Im(F)$ is the smallest subspace of $F$ through which $f$ can be factored. For any subspace $X$ of $E$, we define the direct image as
    \[f(x) = Im(fi_x)\]
    Where $i_X$ is the inclusion of $X$ into $E$.
\end{definition}

\begin{lemma}[(4) Direct Images Transitive]
(Leroy Lemme 4)
    \label{lem:direct_images_transitive}
    \uses{def:direct_images}
    Given two morphisms $f : E \rightarrow F$ and $g : F \rightarrow G$ and a subspace $X$ of $E$, we have
    \[(gf)(X) = g(f(X))\]
\end{lemma}

\begin{lemma}[(5) Direct Images Families]
(Leroy Lemme 5)
    \label{lem:direct_images_families}
    \uses{def:direct_images}
    For all morphisms $f: E \rightarrow F$ and a family $(X_i)$ of subspaces of $E$, the following holds:
    \[f(\cup_i X_i) = \cup_i f(X_i)\]
\end{lemma}

\subsection{Inverse Images}
\begin{definition}[Inverse Images]
    \label{def:inverse_images}
    \uses{lem:direct_images_transitive}
    We have a morphism of spaces $f : E \rightarrow F$ and a subspace $Y$ of $F$. The inverse image $f^{-1}(Y)$ is the biggest subspace $X$ of $E$ such that $f(X) \subset Y$. \\
    More generally for a morphism $h : A \rightarrow E$, the necessary and sufficient condition for $h$ to factor through $f^{-1}$ is that $fh$ factors through $Y$.
    \[Im h \subset f^{-1}(Y)\iff f(Im h) \subset Y \iff Im(fh) \subset Y \] \
\end{definition}

\subsection{Open subspaces}
\begin{definition}[$e_U$]
    TODO bessere benennung
    \label{def:e_U}
    \uses{def:nucleus}
    Let $E$ be a space with $U, H \in O(E)$. We donote by $e_U$ the largest $W \in O(E)$ such that $W \cap U \subset H$. We verify that $e_U$ is the nucleus of a subspace, which we will temporarily denote by $[U]$.
\end{definition}

\begin{lemma}[(6) Open subspaces]
(Leroy Lemma 6)
    \label{lem:open_subspaces}
    \uses{def:e_U}
    For all subspaces $X$ of $E$ and any $U \in O(E)$:
    TODO
\end{lemma}

