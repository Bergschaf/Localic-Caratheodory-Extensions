\begin{lemma}[Frame]
    \label{lem:frame}
    (Already in Mathlib )A Frame can be viewed as a Category.
    There exists a morphism between $A$ and $B$ iff $A \le B$.
    \leanok
\end{lemma}

\begin{definition}[Top -> Frame]
    \label{def:top_frame_func}
    \uses{lem:frame}
    (Brauchen wir das überhaupt?) There exists a contravariant Functor from a topological Space to the corresponding Frame Category with open Sets as objects. $f: X \rightarrow O(X)$
\end{definition}

\begin{definition}[$f^*$ and $f_*$]
    \label{def:f_star}
    \uses{def:top_frame_func}
    \lean{f_obenstern}
    \leanok
    For every continuous Function $f : X \rightarrow Y$ between topological Spaces, there exists a pair of functors $(f^*,f_*)$.
    \begin{gather*}
        f* = f^{-1} : O(Y) \rightarrow O(X)\\
        f_* : O(X) \rightarrow O(Y) \coloneqq A \mapsto \bigcup_{f^*(v) \le A} v\\
    \end{gather*}
\end{definition}

\begin{lemma}[$f^*$ commutes]
    \label{lem:f_star_comm}
    \uses{def:f_star}

    $f^*$ commutes with finite meets and arbitrary joins. This is the same as saying that the Frame Category of open Sets has all small coproducts and all finite Limits. \href{https://ncatlab.org/nlab/show/frame}{nlab}
\end{lemma}

\begin{lemma}[Homsets]
    \label{lem:f_star_homsets}
    \uses{lem:f_star_adj}
    \leanok
    (Already in Mathlib)
    \[Hom_{O(Y)}(f*(V), A) = Hom_{O(x)}(v, f_*(A))\]
\end{lemma}

\begin{lemma}[$f^* \dashv f_*$]
    \label{lem:f_star_adj}
    \uses{def:f_star}
    \lean{f_adj}
    \leanok
    $f^*$ is the right adjoint to $f_*$
\end{lemma}

\begin{lemma}[Embedding]
(Leroy Lemme 1)
    \label{lem:embedding}
    \uses{def:f_star}
    \lean{f_surjective_injective}
    $f: X \rightarrow Y$ is an embedding, if at least one of the following arguemnts holds:
    \begin{enumerate}
        \item $f^*$ is surjective
        \item $f_*$ is injective
        \item $f^{*}f_* = 1_{O(X)})$
    \end{enumerate}
\end{lemma}

\begin{definition}[Nucleus]
    \label{def:nucleus}
    \uses{lem:frame}
    \lean{Nucleus}
    \leanok
    A nucleus is a map $e : O(E) \rightarrow O(E)$ with the following three properties:
    \begin{enumerate}
        \item $e$ is idempotent
        \item $U \le e U$
        \item $e(U \cap V) = e(U) \cap e(V)$
    \end{enumerate}
\end{definition}

\begin{lemma}[Nucleus]
(Leroy Lemme 3)
    \label{lem:nucleus}
    \uses{def:f_star,def:nucleus}
    Let $e : O(E) \rightarrow O(E)$ be a nucleus. Then there is a space $X$ and a morphism $f: X \rightarrow E$ such that $e = f_*f^*$. (The same holds for embeddings ??)
\end{lemma}

\begin{definition}[Subframe]
(Leroy CH 3)
    \label{lem:subframe}
    \uses{lem:nucleus}
    A subspace $Y \subset X$ is defined by a nucleus $e_Y: O(X) \rightarrow O(X)$, such that $O(Y) = Im(e_Y) = \{U \in O(X) | e_Y(U) = U\}$.
    The corresponding embedding (???) is $i_X : O(Y) \rightarrow O(X)$. $i^*_X(V) = e_X(V)$, $(i_X)_*(U) = U$
\end{definition}

\begin{definition}[Subframe Inclusion]
(Stimmt das?)(Leroy Ch 3)
    \label{def:subframe_inclusion}
    \uses{lem:subframe}
    $X \subset Y$ if $e_Y \le e_X$
\end{definition}

\begin{lemma}[factorisation]
(Leroy Lemme 2)
    \label{lem:factorisation}
    \uses{def:f_star}
    Let $i : X \rightarrow E$ be an embedding and $f: Y \rightarrow E$ be a morphism of spaces. To have $f$ factor through $i$, it is necessary and sufficient that $i_*i^*(V) \le f_*f^*(V) for all V \in O(E)$.
\end{lemma}


\begin{lemma}[Familiy of subspaces]
(Leroy CH 4)
    \label{lem:family_of_subspaces}
    \uses{lem:nucleus}
    Let $X_i$ be a family of subspaces of $E$ and $e_i$ be the corresponding nuclei. For every $V \in O(E)$, let $e(V)$ be the onion of all $W \in O(E)$ which are contained in every $e_i(V)$. Then \\
    \begin{enumerate}
        \item $e$ is the corresponding nuclus of a subspace $X$ of $E$
        \item a subspace $Z$ of $E$ conatains $x$ if and only if it containd all $X_i$ x is thus called the union of $X_i$ denoted by $\bigcup_i X_i$
    \end{enumerate}
\end{lemma}