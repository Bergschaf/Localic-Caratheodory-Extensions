\chapter{Leroy}


\section{F star}

\begin{definition}[Frame]
    \label{lem:frame}
    (Already in Mathlib )A Frame can be viewed as a Category.
    There exists a morphism between $A$ and $B$ iff $A \le B$.
    \leanok
\end{definition}

\begin{definition}[Top -> Frame]
    \label{def:top_frame_func}
    \uses{lem:frame}
    (Brauchen wir das überhaupt?) There exists a contravariant Functor from a topological Space to the corresponding Frame Category with open Sets as objects. $f: X \rightarrow O(X)$
\end{definition}

\begin{definition}[$f^*$ and $f_*$]
    \label{def:f_star}
    \uses{def:top_frame_func}
    \lean{f_obenstern}
    \leanok
    For every continuous Function $f : X \rightarrow Y$ between topological Spaces, there exists a pair of functors $(f^*,f_*)$.
    \begin{gather*}
        f* = f^{-1} : O(Y) \rightarrow O(X)\\
        f_* : O(X) \rightarrow O(Y) := A \mapsto \bigcup_{f^*(v) \le A} v\\
    \end{gather*}
\end{definition}

\begin{lemma}[$f^*$ commutes]
    \label{lem:f_star_comm}
    \uses{def:f_star}

    $f^*$ commutes with finite meets and arbitrary joins. This is the same as saying that the Frame Category of open Sets has all small coproducts and all finite Limits. \href{https://ncatlab.org/nlab/show/frame}{nlab}
\end{lemma}

\begin{lemma}[Homsets]
    \label{lem:f_star_homsets}
    \uses{lem:f_star_adj}
    \leanok
    (Already in Mathlib)
    \[Hom_{O(Y)}(f*(V), A) = Hom_{O(x)}(v, f_*(A))\]
\end{lemma}

\begin{lemma}[$f^* \dashv f_*$]
    \label{lem:f_star_adj}
    \uses{def:f_star}
    \lean{f_adj}
    \leanok
    $f^*$ is the right adjoint to $f_*$
\end{lemma}
\begin{proof}
    \leanok
\end{proof}

\begin{lemma}[triangle]
(Mclane p. 485)

    The triangular identities reduce to the following equalities:
    \[f^*f_*f^* = f^* \quad \text{and} \quad f_*f^*f_* = f_*\]
    \lean{triangle_one}
    \leanok
    \label{lem:triangle}
\end{lemma}
\begin{proof}
    \uses{lem:f_star_adj}

    This follows from the triangular identities of the adjunction.
    \leanok
\end{proof}


\section{Embedding}

\begin{lemma}[Embedding]
(Leroy Lemme 1)
    \label{lem:embedding}
    \uses{def:f_star,lem:triangle}
    \lean{f_injective_surjective}
    The following arguments are equivalent:
    \begin{enumerate}
        \item $f^*$ is surjective
        \item $f_*$ is injective
        \item $f^{*}f_* = 1_{O(X)})$
    \end{enumerate}
\end{lemma}
\begin{proof}
    This follows from the triangular identities. \leanok
\end{proof}

\begin{definition}[Embedding]
    \label{def:embedding}
    \uses{lem:embedding}
    An embedding is a morphism that satisfies the conditions of \cref{lem:embedding}
    \lean{Leroy_Embedding}
    \leanok
\end{definition}


\section{Subspaces}

\begin{definition}[Nucleus]
    \label{def:nucleus}
    \uses{lem:frame}
    \lean{Nucleus}
    \leanok
    A nucleus is a map $e : O(E) \rightarrow O(E)$ with the following three properties:
    \begin{enumerate}
        \item $e$ is idempotent
        \item $U \le e U$
        \item $e(U \cap V) = e(U) \cap e(V)$
    \end{enumerate}
\end{definition}

\begin{lemma}[Nucleus]
(Leroy Lemme 3)
    \label{lem:nucleus}
    \uses{def:f_star,def:nucleus,def:embedding}
    Let $e : O(E) \rightarrow O(E)$ be a nucleus. Then there is a space $X$ and a morphism $f: X \rightarrow E$ such that $e = f_*f^*$. (The same holds for embeddings ??)
\end{lemma}

\begin{definition}[Subframe]
(Leroy CH 3)
    \label{lem:subframe}
    \uses{lem:nucleus}
    A subspace $Y \subset X$ is defined by a nucleus $e_Y: O(X) \rightarrow O(X)$, such that $O(Y) = Im(e_Y) = \{U \in O(X) | e_Y(U) = U\}$.
    The corresponding embedding (???) is $i_X : O(Y) \rightarrow O(X)$. $i^*_X(V) = e_X(V)$, $(i_X)_*(U) = U$

    And every nucleus $e$ on $O(X)$ defines a subspace $Y$ of $X$ by $O(Y) = Im(e)$
\end{definition}

\begin{definition}[Subframe Inclusion]
(Stimmt das?)(Leroy Ch 3)
    \label{def:subframe_inclusion}
    \uses{lem:subframe}
    $X \subset Y$ if $e_Y \le e_X$
\end{definition}

\begin{lemma}[factorisation]
(Leroy Lemme 2)
    \label{lem:factorisation}
    \uses{def:f_star, def:embedding}
    Let $i : X \rightarrow E$ be an embedding and $f: Y \rightarrow E$ be a morphism of spaces. To have $f$ factor through $i$, it is necessary and sufficient that $i_*i^*(V) \le f_*f^*(V) for all V \in O(E)$.
\end{lemma}


\begin{lemma}[Familiy of subspaces]
(Leroy CH 4)
    \label{lem:family_of_subspaces}
    \uses{lem:nucleus}
    Let $X_i$ be a family of subspaces of $E$ and $e_i$ be the corresponding nuclei. For every $V \in O(E)$, let $e(V)$ be the onion of all $W \in O(E)$ which are contained in every $e_i(V)$. Then \\
    \begin{enumerate}
        \item $e$ is the corresponding nucleus of a subspace $X$ of $E$
        \item a subspace $Z$ of $E$ contains $x$ if and only if it contains all $X_i$ x is thus called the union of $X_i$ denoted by $\bigcup_i X_i$
    \end{enumerate}
\end{lemma}

\subsection{Direct Images}
\begin{definition}[Direct Images]
    \uses{lem:subframe}
    \label{def:direct_images}
    Let $f : E \rightarrow F$ be a morphism of spaces. The map $f_*f^* : O(F) \to O(F)$ is the nucleus of the subspace $Im(f)$ of $F$. By (lemma 2), $Im(F)$ is the smallest subspace of $F$ through which $f$ can be factored. For any subspace $X$ of $E$, we define the direct image as
    \[f(x) = Im(fi_x)\]
    Where $i_X$ is the inclusion of $X$ into $E$.
\end{definition}

\begin{lemma}[(4) Direct Images Transitive]
(Leroy Lemme 4)
    \label{lem:direct_images_transitive}
    \uses{def:direct_images}
    Given two morphisms $f : E \rightarrow F$ and $g : F \rightarrow G$ and a subspace $X$ of $E$, we have
    \[(gf)(X) = g(f(X))\]
\end{lemma}

\begin{lemma}[(5) Direct Images Families]
(Leroy Lemme 5)
    \label{lem:direct_images_families}
    \uses{def:direct_images}
    For all morphisms $f: E \rightarrow F$ and a family $(X_i)$ of subspaces of $E$, the following holds:
    \[f(\cup_i X_i) = \cup_i f(X_i)\]
\end{lemma}

\subsection{Inverse Images}
\begin{definition}[Inverse Images]
    \label{def:inverse_images}
    \uses{lem:direct_images_transitive}
    We have a morphism of spaces $f : E \rightarrow F$ and a subspace $Y$ of $F$. The inverse image $f^{-1}(Y)$ is the biggest subspace $X$ of $E$ such that $f(X) \subset Y$. \\
    More generally for a morphism $h : A \rightarrow E$, the necessary and sufficient condition for $h$ to factor through $f^{-1}$ is that $fh$ factors through $Y$.
    \[Im h \subset f^{-1}(Y)\iff f(Im h) \subset Y \iff Im(fh) \subset Y \] \
\end{definition}

\subsection{Open subspaces}
\begin{definition}[$e_U$]
    TODO bessere benennung
    \label{def:e_U}
    \uses{def:nucleus}
    Let $E$ be a space with $U, H \in O(E)$. We donote by $e_U$ the largest $W \in O(E)$ such that $W \cap U \subset H$. We verify that $e_U$ is the nucleus of a subspace, which we will temporarily denote by $[U]$.
\end{definition}

\begin{lemma}[(6a) Open subspaces]
(Leroy Lemma 6a)
    \label{lem:open_subspaces_a}
    \uses{def:e_U}
    For all subspaces $X$ of $E$ and any $U \in O(E)$:
    \[X \subset [U] \iff e_X(U) = 1_E\]
\end{lemma}
\begin{proof}

\end{proof}

\begin{lemma}[(6b) Open subspaces]
(Leroy Lemma 6b)
    For all $U, V \in O(E)$, we have:
    \[[U \cap V] = [U] \cap [V]\]
    \[e_{U \cap V} = e_Ue_V=e_Ve_U\]
    \[U \subset V \iff [U] \subset [V]\]
    \label{lem:open_subspaces_b}
    \uses{def:e_U}
\end{lemma}
\begin{proof}
\end{proof}

\begin{lemma}[(6c) Open subspaces]
    For all families $V_i$ of elements of $O(E)$, we have:
    \[\cup_i[V_i] = [\cup_iV_i]\]

    \label{lem:open_subspaces_c}
    \uses{def:e_U}
\end{lemma}

\begin{lemma}[(6d) Open subspaces]
    For all morphisms of spaces $f: E \to F$ and all $V \in O(E)$, we have:
    \[f^{-1}([V])] = [f^*(V)]\]
    \label{lem:open_subspaces_d}
    \uses{def:e_U}
\end{lemma}

\begin{lemma}[(7) Open subspaces]
    Let $X$ be a subspace of $E$ and $U \in O(E)$. For all $V \in O(E)$, we have:
    \[V \subset e_X(U) \iff [V] \cap X \subset [U]\]
    \label{lem:open_subspaces_7}
    \uses{def:e_U}
\end{lemma}


\section{Leroy Chapter II}\label{sec:leroy-chapter-ii}

\begin{lemma}[Union of Intersections]
(Leroy Lemme 2.4)
    \label{lem:union_of_intersections}
    \uses{def:intersection}
    For any family $B_i$ of sublocals of a local $E$ and a sublocal $A$, we have:
    \[A \cup (\bigcap_i B_i) = \bigcap_i (A \cap B_i)\]
    (TODO hier weitermachen)
\end{lemma}


\section{Leroy Chapter III}\label{sec:leroy-chapter-iii}
\begin{definition}[Measure on Locals]
    \label{def:measure_on_locals}
    (Leroy III.1.)
\end{definition}

\begin{definition}[Caratheodory]
    \label{def:caratheodory}
\end{definition}

\begin{lemma}[Proptery 0 (Commutes with sup)]
(Leroy lemme 3.1)
    \label{lem:commutes_with_sup}
    \uses{def:measure_on_locals,def:caratheodory}
    The caratheodory extension of a measure on a local commutes with unions of increasing families.
    (Senf von noa: commutes with filtered colimits)
\end{lemma}



\begin{definition}[Regular Local]
    \label{def:regular_local}
\end{definition}

\begin{definition}[Open sublocal]
    \label{def:open_sublocal}

\end{definition}

\begin{definition}[Neighborhood]
    \label{def:neighborhood}
    \uses{def:open_sublocal}
    A neighborhood of a sublocal $A$ of $X$ is an open sublocal $V$ of $X$ such that $A \le V$.
\end{definition}

\begin{lemma}[Regularity of Sublocals]
(Leroy lemme 3.2)
    \label{lem:regularity_of_sublocals}
    \uses{def:regular_local, def:neighborhood}
    In a regular local, any sublocal is regular, meaning that it is the intersection of all open neighborhoods.
\end{lemma}

\begin{definition}[Complement]
    \label{def:complement}
    \uses{def:open_sublocal}
    The complement of an open sublocal $U$ of $X$ is the sublocal $X \setminus U$.
    (Leroy p. 12) (+ Senf brauchen wir das allgemein??)
\end{definition}

\begin{lemma}[Property 1]
(Leroy Lemme 3.3)
    \label{lem:property_1}
    \uses{def:complement, lem:commutes_with_sup, lem:regularity_of_sublocals}
    For any open sublocal $U$ of a local $X$, the caratheodory extension of a measure on $X$ satisfies \[\mu(U) + \mu(X \setminus U) = \mu(X)\]
\end{lemma}

\begin{definition}[Intersection]
    \label{def:intersection}
    The intersection of any Family $A_i$ of sublocals of a local $E$ is defined as ........ (+ Senf pullback)
\end{definition}

\begin{proposition}[Sublocals structure]
    Obacht TODO
    \label{prop:sublocals_structure}
    \uses{def:intersection,def:complement,lem:union_of_intersections}
    The sublocals of a local $E$ form a complete (distributive vlt.) lattice with the operations of intersection and complement.
\end{proposition}


\begin{lemma}[Property 2]
(Leroy Lemm 3.4)
    \label{lem:property_2}
    \uses{def:complement,def:intersection,lem:property_1}
    For any open sublocal $U$ and any sublocal $A$ of a local $E$, the caratheodory extension of a measure on $X$ satisfies \[\mu(A) = \mu(A \cap U) + \mu(A \cap(E\setminus U))\]
\end{lemma}

\begin{lemma}[Property 3]
(Leroy Lemm 3.5)
    \label{lem:property_3}
    \uses{lem:property_2,lem:commutes_with_sup}
    For a increasing family $V_{\alpha}$ of open sublocals of $E$ and any sublocal $A$, we have:
    \[\mu(A \cap(\bigcup V_{\alpha})) = \sup_\alpha \mu(A\cap V_\alpha)\]
\end{lemma}


\begin{lemma}[Commutes with inf opens]
(Leroy Lemme 3.6)
    \label{lem:commutes_with_inf_opens}
    \uses{lem:property_1,prop:sublocals_structure}
    For any measure on a local $X$ and a decreasing family $V_i$ of open sublocals, the caratheodory extension fulfills: $\mu (\inf V_i) = \inf \mu(V_i)$.
\end{lemma}

\begin{lemma}[Caratheodory Extensions are monotonic]
    \label{lem:monotonic}
    \uses{def:measure_on_locals,def:caratheodory} TODO uses inlcusion of sublocals
    The caratheodory extension is monotonic i.e. \[A \le B \implies \mu (A) \le \mu (B)\]
\end{lemma}

\begin{proposition}[Elementary Properties of Caratheodory Extensions]
(Leroy lemme 3.3, 3.4, Corollary 3.1, Lemme 3.5)
    \label{prop:elementary_properties_of_caratheodory_extensions}
    \uses{lem:property_1,lem:property_2,lem:commutes_with_sup, lem:property_3, lem:commutes_with_inf_opens,lem:monotonic}
    For any measure on a local $X$, the caratheodory extension satisfies the following properties:
    \begin{enumerate}
        \item It is monotonic i.e. \[A \le B \implies \mu (A) \le \mu (B)\]
        \item Commutes with unions of increasing families
        \item $\mu(U) + \mu(X \setminus U) = \mu(X)$
        \item $\mu(A) = \mu(A \cap U) + \mu(A \cap(E\setminus U))$
        \item For a increasing family $V_{\alpha}$ of open sublocals of $E$ and any sublocal $A$, we have:
        \[\mu(A \cap(\bigcup V_{\alpha})) = \sup_\alpha \mu(A\cap V_\alpha)\]
        \item For any measure on a local $X$ and a decreasing family $V_i$ of open sublocals, the caratheodory extension fulfills: $\mu (\inf V_i) = \inf \mu(V_i)$.
    \end{enumerate}
    In particular, for two open sublocals $U$ and $V$ of $X$ and any sublocal $A$ of $X$, we have
    \[\mu(A \cap (U \cup V)) = \mu(A\cap U) + \mu(A\cap V) - \mu(A \cap U \cap V)\]
\end{proposition}



\begin{proposition}[strictly additve]
(Leroy theorem 3.3.1)
    \label{prop:strictly_additive}
    \uses{prop:commutes_with_inf,lem:regularity_of_sublocals}
    For any measure on a local $X$, the caratheodory extension is
    strictly additive i.e. $\mu (A \cup B) = \mu(A) + \mu(B) - \mu(A \cap B)$
\end{proposition}


\begin{proposition}[reductive]
    \label{prop:reductive}  (Proposition 3.3.1)
    \uses{lem:subframe, prop:elementary_properties_of_caratheodory_extensions}
    For any measure on a local $X$, the caratheodory extension is
    reductive i.e. for all $A \le X$ the set $\{A' \subset A, \mu(A') = \mu(A)\}$ has a minimal element
\end{proposition}


\begin{proposition}[Commutes with inf]
(Leroy lemme 3.7 et principal)
    \label{prop:commutes_with_inf}
    \uses{prop:reductive,lem:commutes_with_inf_opens,lem:regularity_of_sublocals,prop:elementary_properties_of_caratheodory_extensions}
    For any measure on a local $X$, the caratheodory extension is
    regular $\mu (\inf A_i) = \inf \mu(A_i)$. For decreasing families $A_i$
\end{proposition}

\begin{theorem}[Main Theorem (very important)]
    \label{thm:main}
    \uses{prop:strictly_additive,prop:commutes_with_inf,prop:reductive}
    For any measure on a local $X$, the caratheodory extension is
    \begin{enumerate}
        \item strictly additive i.e. $\mu (A \cup B) = \mu(A) + \mu(B) - \mu(A \cap B)$
        \item commutes with inf $\mu (\inf A_i) = \inf \mu(A_i)$
        \item reductive i.e. for all $A \le X$ the set $\{A' \subset A, \mu(A') = \mu(A)\}$ has a minimal element
    \end{enumerate}
\end{theorem}


\section{Leroy Chapter V}\label{sec:leroy-chapter-v}

\begin{lemma}[Opens]
(Leroy V.1 Remarque 2)
    \label{lem:opens_correspond}
    \uses{def:open_sublocal}
    The Open subsets of any good enough topological space correspond precisely to the open sublocals of the corresponding local.
\end{lemma}

\begin{lemma}[Subset Sublocal]
(leroy V.1 Remarque 3)
    \label{lem:subset_sublocal}
    \uses{def:good_enough_topological_space}
    Any subset $X$ of a good enough topological space $E$ induces a sublocal $[X]$ of the corresponding local. This is an order preserving embedding.
\end{lemma}

\begin{definition}[Good enough topological space]
    \label{def:good_enough_topological_space}
    blackbox to mathlib???? )
\end{definition}

\begin{lemma}[Subset to sublocal Part 1]
(Leroy Proposition 5.1.1)
    \label{lem:subset_to_sublocal_part_1}
    \uses{lem:opens_correspond,lem:subset_sublocal,def:good_enough_topological_space}

    For two subspaces $X$ and $Y$ of $E$ and an open subspaces $U$ of $E$, we have:
    \begin{enumerate}
        \item $X \subset Y \implues [X] \subset [Y]$
        \item $X \subset U \iff [X] \subset [U]$
        \item If $E$ is a good enough topological space, then \[X \subset Y \iff [X] \subset [Y]\]
    \end{enumerate}
\end{lemma}

\begin{lemma}[Subset to sublocal Part 2]
(Leroy Proposition 5.1.2)
    \label{lem:subset_to_sublocal_part_2}
    \uses obacht
    For an open subspace $U$ of $E$ and a subspace $X$ of $E$, we have:
    \[[U \cap X] = [U] \cap [X]\]
\end{lemma}

\begin{proposition}[Subset to sublocal preserves structure]
    \label{prop:subset_to_sublocal_structure}
    \uses{lem:subset_to_sublocal_part_1,lem:subset_to_sublocal_part_2}
    For two subspaces $X$ and $Y$ of $E$ and an open subspaces $U$ of $E$, we have:
    \begin{enumerate}
        \item $X \subset Y \implues [X] \subset [Y]$
        \item $X \subset U \iff [X] \subset [U]$
        \item If $E$ is a good enough topological space, then \[X \subset Y \iff [X] \subset [Y]\]
        \item \[[U \cap X] = [U] \cap [X]\]
    \end{enumerate}

\end{proposition}

\begin{theorem}[Measure top to loc]
    \label{thm:measure_top_to_loc}
    \uses{lem:opens_correspond,lem:subset_sublocal}
    Any measure on a good enough topological space $X$ induces a measure on the corresponding local. Furthermore, the classical caratheodory extension onto $\mathcal{P}(X)$ agrees with the restriction of the caratheodory extension of the induced measure on the local.
\end{theorem}

\begin{corollary}[Goal]
    \label{cor:goal}
    \uses{thm:measure_top_to_loc,thm:main}
    One can interpret any classical borel measure as a measure on locals and their life is good :)
\end{corollary}




