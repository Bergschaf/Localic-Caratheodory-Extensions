\chapter{Leroy Chapter I}


\section{F star}

\begin{definition}[$f^*$ and $f_*$]
    \label{def:f_star}
    \lean{f_obenstern}
    \leanok
    For every continuous Function $f : X \rightarrow Y$ between topological Spaces, there exists a pair of functors $(f^*,f_*)$.
    \begin{gather*}
        f* = f^{-1} : O(Y) \rightarrow O(X)\\
        f_* : O(X) \rightarrow O(Y) := A \mapsto \bigcup_{f^*(v) \le A} v\\
    \end{gather*}
\end{definition}

\begin{lemma}[$f^*$ commutes]
    \label{lem:f_star_comm}
    \uses{def:f_star}
    $f^*$ commutes with finite meets and arbitrary joins. This is the same as saying that the Frame Category of open Sets has all small coproducts and all finite Limits. \href{https://ncatlab.org/nlab/show/frame}{nlab}
\end{lemma}


\begin{lemma}[$f^* \dashv f_*$]
    \label{lem:f_star_adj}
    \uses{def:f_star}
    \lean{f_adj}
    \leanok
    $f^*$ is the right adjoint to $f_*$
\end{lemma}
\begin{proof}
    \leanok
\end{proof}

\begin{lemma}[triangle]
(Mclane p. 485)

    The triangular identities reduce to the following equalities:
    \[f^*f_*f^* = f^* \quad \text{and} \quad f_*f^*f_* = f_*\]
    \lean{triangle_one}
    \leanok
    \label{lem:triangle}
\end{lemma}
\begin{proof}
    \uses{lem:f_star_adj}
    This follows from the triangular identities of the adjunction.
    \leanok
\end{proof}


\section{Embedding}
\begin{lemma}[Embedding]
(Leroy Lemme 1)
    \label{lem:embedding}
    \uses{def:f_star,lem:triangle}
    \lean{f_injective_surjective}
    \leanok
    The following arguments are equivalent:
    \begin{enumerate}
        \item $f^*$ is surjective
        \item $f_*$ is injective
        \item $f^{*}f_* = 1_{O(X)})$
    \end{enumerate}
\end{lemma}
\begin{proof}
    This follows from the triangular identities. \leanok
\end{proof}

\begin{definition}[Embedding]
    \label{def:embedding}
    \uses{lem:embedding}
    An embedding is a morphism that satisfies the conditions of \cref{lem:embedding}
    \lean{Leroy_Embedding}
    \leanok
\end{definition}


\section{Sublocals}
\begin{definition}[Nucleus]
    \label{def:nucleus}
    \uses{lem:frame}
    \lean{Nucleus}
    \leanok
    A nucleus is a map $e : O(E) \rightarrow O(E)$ with the following three properties:
    \begin{enumerate}
        \item $e$ is idempotent
        \item $U \le e U$
        \item $e(U \cap V) = e(U) \cap e(V)$
    \end{enumerate}
\end{definition}

\begin{lemma}[Nucleus]
(Leroy Lemme 3)
    \label{lem:nucleus}
    \lean{nucleus_equiv_subframe_3}
    \leanok
    \uses{def:f_star,def:nucleus,def:embedding}
    Let $e : O(E) \rightarrow O(E)$ be monotonic. The following are equivalent:
    \begin{enumerate}
        \item $e$ is a nucleus
        \item There is a locale $X$ and a morphism $f: X \rightarrow E$ such that $e = f_*f^*$.
        \item Then there is a locale $X$ and a embedding $f: X \rightarrow E$ such that $e = f_*f^*$.
    \end{enumerate}
\end{lemma}
\begin{proof}
    \leanok
\end{proof}

\begin{definition}[Subframe]
(Leroy CH 3)
    \label{def:subframe}
    \uses{lem:nucleus}
    \lean{Subframe}
    \leanok
    A sublocal $Y \subset X$ is defined by a nucleus $e_Y: O(X) \rightarrow O(X)$, such that $O(Y) = Im(e_Y) = \{U \in O(X) | e_Y(U) = U\}$.
    The corresponding embedding (???) is $i_X : O(Y) \rightarrow O(X)$. $i^*_X(V) = e_X(V)$, $(i_X)_*(U) = U$

    And every nucleus $e$ on $O(X)$ defines a sublocal $Y$ of $X$ by $O(Y) = Im(e)$
\end{definition}

\begin{definition}[Subframe Inclusion]
(Stimmt das?)(Leroy Ch 3)
    \label{def:subframe_inclusion}
    \uses{def:subframe}
    \lean{subframe_inclusion}
    \leanok
    $X \subset Y$ if $e_Y \le e_X$
\end{definition}

\begin{lemma}[factorisation]
(Leroy Lemme 2)
    \label{lem:factorisation}
    \uses{def:f_star, def:embedding}
    \lean{factorisation}
    \leanok
    Let $i : X \rightarrow E$ be an embedding and $f: Y \rightarrow E$ be a morphism of spaces. To have $f$ factor through $i$, it is necessary and sufficient that $i_*i^*(V) \le f_*f^*(V) for all V \in O(E)$.
\end{lemma}
\begin{proof}
    \leanok
\end{proof}

\subsection{(1.4) Subframe Union and Intersection}
\begin{definition}[Subframe Union]
(Leroy CH 1.4)
    \label{def:subframe_union}
    \uses{def:subframe_inclusion,lem:nucleus}
    \lean{Nucleus_max}
    \leanok
    Let $(X_i)_i$ be a family of subframes of $E$ and $(e_i)_i$ the corresponding nuclei.
    For all $V \in O(E)$, let $e(V)$ be the union of all $W \in O(E)$ which are contained in all $e_i(V)$.
\end{definition}

\begin{lemma}[Familiy of Subframes]
(Leroy CH 4)
    \label{lem:family_of_subframes}
    \uses{lem:nucleus, def:subframe_union}
    Let $X_i$ be a family of subframes of $E$ and $e_i$ be the corresponding nuclei. For every $V \in O(E)$, let
    $e(V)$ be the union of all $W \in O(E)$ which are contained in every (TODO wieso every) $e_i(V)$. Then \\
    \begin{enumerate}
        \item $e$ is the corresponding nucleus of a subframe $X$ of $E$
        \item a subframe $Z$ of $E$ contains $x$ if and only if it contains all $X_i$. $X$ is thus called the union of
        $X_i$ denoted by $\bigcup_i X_i$
    \end{enumerate}
\end{lemma}

\begin{definition}[Subframe intersection]
    \label{def:subframe_intersection}
    \uses{def:subframe_union,lem:nucleus}
    \lean{Nucleus_min}
    \leanok
    Let $(X_i)_i$ be a family of subframes of $E$ and $(e_i)_i$ the corresponding nuclei.
    For all $V \in O(E)$, the intersection $\bigcap X_i$ of all.
\end{definition}

\subsection{(1.5) Direct Images}

\begin{definition}[Direct Images]
    \uses{def:subframe}
    \label{def:direct_images,lem:factorisation, def:subframe_intersection}
    Let $f : E \rightarrow F$ be a morphism of Frames. The map $f_*f^* : O(F) \to O(F)$ is the nucleus of the subframe
    $Im(f)$ of $F$. By (lemma 2), $Im(F)$ is the smallest subframe of $F$ through which $f$ can be factored. For any subframe $X$ of $E$, we define the direct image as
    \[f(x) = Im(fi_x)\]
    Where $i_X$ is the inclusion of $X$ into $E$.
\end{definition}

\begin{lemma}[(4) Direct Images Transitive]
(Leroy Lemme 4)
    \label{lem:direct_images_transitive}
    \uses{def:direct_images}
    Given two morphisms $f : E \rightarrow F$ and $g : F \rightarrow G$ and a subspace $X$ of $E$, we have
    \[(gf)(X) = g(f(X))\]
\end{lemma}

\begin{lemma}[(5) Direct Images Families]
(Leroy Lemme 5)
    \label{lem:direct_images_families}
    \uses{def:direct_images, lem:family_of_subframes}
    For all morphisms $f: E \rightarrow F$ and a family $(X_i)$ of subspaces of $E$, the following holds:
    \[f(\cup_i X_i) = \cup_i f(X_i)\]
\end{lemma}

\subsection{(6) Inverse Images}
\begin{definition}[Inverse Images]
    \label{def:inverse_images}
    \uses{lem:direct_images_transitive}
    We have a morphism of spaces $f : E \rightarrow F$ and a subspace $Y$ of $F$. The inverse image $f^{-1}(Y)$ is the biggest subspace $X$ of $E$ such that $f(X) \subset Y$. \\
    More generally for a morphism $h : A \rightarrow E$, the necessary and sufficient condition for $h$ to factor through $f^{-1}$ is that $fh$ factors through $Y$.
    \[Im h \subset f^{-1}(Y)\iff f(Im h) \subset Y \iff Im(fh) \subset Y \] \
\end{definition}

\subsection{(7) Open Sublocals}
\begin{definition}[$e_U$]
    \label{def:e_U}
    \uses{def:nucleus,def:subframe,def:subframe_union, def:subframe_intersection}
    \lean{e_U}
    \leanok
    Let $E$ be a space with $U, H \in O(E)$. We donote by $e_U$ the largest $W \in O(E)$ such that $W \cap U \subset H$. We verify that $e_U$ is the nucleus of a subspace, which we will temporarily denote by $[U]$.
\end{definition}


\begin{definition}[Open sublocal]
    \label{def:open_sublocal}
    \uses{def:e_U}
        \lean{is_open}
    \leanok
    For any $U \in O(E)$, the sublocal $[U]$ is called an open sublocal of $E$.
    (+Senf: stimmt das mit dem üblichen überein???)
\end{definition}


\begin{lemma}[(6,7) Open subspaces]
(Leroy Lemma 6,7)
    \label{lem:sublocal_properties}
    \uses{def:e_U,def:open_sublocal}
    \begin{enumerate}
        \item For all subspaces $X$ of $E$ and any $U \in O(E)$:
        \[X \subset [U] \iff e_X(U) = 1_E\]

        \item For all $U, V \in O(E)$, we have:
        \[[U \cap V] = [U] \cap [V]\]
        \[e_{U \cap V} = e_Ue_V=e_Ve_U\]
        \[U \subset V \iff [U] \subset [V]\]
        \item
        For all families $V_i$ of elements of $O(E)$, we have:
        \[\cup_i[V_i] = [\cup_iV_i]\]

        \item
        For all morphisms of spaces $f: E \to F$ and all $V \in O(E)$, we have:
        \[f^{-1}([V])] = [f^*(V)]\]
        \item
        Let $X$ be a subspace of $E$ and $U \in O(E)$. For all $V \in O(E)$, we have:
        \[V \subset e_X(U) \iff [V] \cap X \subset [U]\]
    \end{enumerate}
\end{lemma}


\begin{definition}[Complement]
    \label{def:complement}
    \uses{def:open_sublocal}
    The complement of an open sublocal $U$ of $X$ is the sublocal $X \setminus U$.
    (Leroy p. 12) (+ Senf brauchen wir das allgemein??)
\end{definition}

\begin{lemma}[(1.8) Properties of Complements]
    \label{lem:properties_of_complements}
    \uses{def:complement}
    For any any open sublocal $V$ of $E$ and any sublocal $X$ of $E$, we have:
    \[V \cup X = E \iff E \setminus V subset X\]
    \[V \cap X = \emptyset \iff X \subset E \setminus V\]
    And thereby:
    \[(E - U = E - V) \implies U = V\]
\end{lemma}

\begin{lemma}[(1.9) Preimage of complements]
    \label{lem:preimage_of_complements}
    \uses{def:complement,lem:properties_of_complements}
    For any morphism of spaces $g: A \to E$ and any open sublocal $V$ of $F$, we have:
    \[g^{-1}(E - V) = A - g^{-1}(V)\]
\end{lemma}

\begin{lemma}[(1.8bis) Properties of Complements Part 2]
    \label{lem:properties_of_complements_part_2}
    \uses{def:complement, lem:preimage_of_complements}
    For any open sublocal $V$ of $E$ and any sublocal $X$ of $E$, we have:
    \[V \cup (E - V) = E \iff V \subset X \]
    \[V \cap (E - V) = \emptyset X \subset V\]
\end{lemma}


\begin{lemma}[(1.10) Nucleus and Intersection]
    \label{lem:nucleus_and_intersection}
    \uses{def:nucleus,def:e_U}
    For any $U \in O(E)$, and sub local $X$ of $E$ we have:
    \[e_{U \cap X} = e_Ue_X\]
    And for a closed $F$
    \[e_{X \cap F} = e_Xe_F\]
\end{lemma}

\begin{lemma}[(1.11) Distribution of Intersections over Unions]
    \label{lem:distribution_of_intersections_over_unions}
    \uses{lem:nucleus_and_intersection}
    Let $X, Y, L$ be three sub locals of $E$. If $L$ is open or closed, we have:
    \[L \cap (X \cap Y) = (L \cap X) \cup (L \cap Y)\]
\end{lemma}

\begin{definition}[Further Topology]
    \label{def:further_topology}
    \begin{enumerate}
        \item $Int X$ is the largest open sublocal contained in $X$
        \item $Ext X$ is the largest open sublocal contained in $E \setminus X$
        \item $\bar{X}$ is the smallest closed sublocal containing $X$
        \item $\partial X = \bar{X} \cap (E - Int X)$
    \end{enumerate}
\end{definition}

%\begin{definition}[gamma]
%    \label{def:gamma}
%    $\gamma(E)$ is the minimal element of all dense sublocals of $E$.
%\end{definition}
