
\section{Basics of Locales}

\begin{definition}[$f^*$ and $f_*$]
    \label{def:f_star}
    \lean{f_obenstern}
    \leanok
    For every continuous Function $f : X \rightarrow Y$ between topological Spaces, there exists a pair of functors $(f^*,f_*)$.
    \begin{gather*}
        f* = f^{-1} : O(Y) \rightarrow O(X)\\
        f_* : O(X) \rightarrow O(Y) := A \mapsto \bigcup_{f^*(v) \le A} v\\
    \end{gather*}
\end{definition}

%\begin{lemma}[$f^*$ commutes]
%    \label{lem:f_star_comm}
%    \uses{def:f_star}
%    $f^*$ commutes with finite meets and arbitrary joins. This is the same as saying that the Frame Category of open
%%Sets has all small coproducts and all finite Limits. \href{https://ncatlab.org/nlab/show/frame}{nlab}
%\end{lemma}


\begin{lemma}[$f^* \dashv f_*$]
    \label{lem:f_star_adj}
    \uses{def:f_star}
    \lean{f_adj}
    \leanok
    $f^*$ is the right adjoint to $f_*$
\end{lemma}
\begin{proof}
    \leanok
\end{proof}

\begin{lemma}[triangle]
(Mclane p. 485)

    The triangular identities reduce to the following equalities:
    \[f^*f_*f^* = f^* \quad \text{and} \quad f_*f^*f_* = f_*\]
    \lean{triangle_one}
    \leanok
    \label{lem:triangle}
\end{lemma}
\begin{proof}
    \uses{lem:f_star_adj}
    This follows from the triangular identities of the adjunction.
    \leanok
\end{proof}


\section{Embedding}
\begin{lemma}[Embedding]
(Leroy Lemme 1)
    \label{lem:embedding}
    \uses{def:f_star,lem:triangle}
    \lean{f_injective_surjective}
    \leanok
    The following arguments are equivalent:
    \begin{enumerate}
        \item $f^*$ is surjective
        \item $f_*$ is injective
        \item $f^{*}f_* = 1_{O(X)})$
    \end{enumerate}
\end{lemma}
\begin{proof}
    This follows from the triangular identities. \leanok
\end{proof}

\begin{definition}[Embedding]
    \label{def:embedding}
    \uses{lem:embedding}
    An embedding is a morphism that satisfies the conditions of \cref{lem:embedding}
    \lean{Leroy_Embedding}
    \leanok
\end{definition}


\section{Sublocals}
\begin{definition}[Nucleus]
    \label{def:nucleus}
    \lean{Nucleus}
    \mathlibok
    A nucleus is a map $e : O(E) \rightarrow O(E)$ with the following three properties:
    \begin{enumerate}
        \item $e$ is idempotent
        \item $U \le e U$
        \item $e(U \cap V) = e(U) \cap e(V)$
    \end{enumerate}
\end{definition}

\begin{lemma}[Nucleus]
(Leroy Lemme 3)
    \label{lem:nucleus}
    %TODO \lean{nucleus_equiv_subframe_1
    \leanok
    \uses{def:f_star,def:nucleus,def:embedding}
    Let $e : O(E) \rightarrow O(E)$ be monotonic. The following are equivalent:
    \begin{enumerate}
        \item $e$ is a nucleus
        \item There is a locale $X$ and a morphism $f: X \rightarrow E$ such that $e = f_*f^*$.
        \item Then there is a locale $X$ and a embedding $f: X \rightarrow E$ such that $e = f_*f^*$.
    \end{enumerate}
\end{lemma}
\begin{proof}
    \leanok
\end{proof}

\begin{definition}[Nucleus Partial Order]
    \label{def:nucleus_partial_order}
    \uses{def:nucleus}
    \lean{Nucleus.instPartialOrder}
    \mathlibok
    \leanok
    For two nuclei $e$ and $f$ on $O(E)$, we say that $e \le f$ if $e(U) \le f(U)$ for all $U \in O(E)$.
    This relation is a partial order.
\end{definition}

\begin{lemma} [Nucleus Intersection]
    \label{lem:nucleus_intersection}
    \mathlibok
    \uses{def:nucleus,def:nucleus_partial_order}
    \lean{Nucleus.instCompleteSemilatticeInf}
    \leanok
    TODO Quelle StoneSpaces S.51
    For a set $S$ of nuclei, the intersection $\bigcap S$ can be computed by $\bigcap S(a) = \bigcap \{j(a) | j \in S \}$.
    This function satisfies the properties of a nucleus and of an infimum.
\end{lemma}
\begin{proof}
    \leanok
\end{proof}


\begin{definition}[Sublocal]
(Leroy CH 3)
    \label{def:sublocal}
    \uses{lem:nucleus}
    % TODO \lean{nucleus_frameHom}
    \leanok
    A sublocal $Y \subset X$ is defined by a nucleus $e_Y: O(X) \rightarrow O(X)$, such that $O(Y) = Im(e_Y) = \{U \in O(X) | e_Y(U) = U\}$.
    The corresponding embedding is $i_X : O(Y) \rightarrow O(X)$. $i^*_X(V) = e_X(V)$, $(i_X)_*(U) = U$
    And every nucleus $e$ on $O(X)$ defines a sublocal $Y$ of $X$ by $O(Y) = Im(e)$
\end{definition}

\begin{definition}[Sublocal Inclusion]
(Stimmt das?)(Leroy Ch 3)
    \label{def:sublocal_inclusion}
    \uses{def:sublocal, def:nucleus_partial_order}
    \lean{Sublocale.le_iff}
    \leanok
    $X \subset Y$ if $e_Y(u) \le e_X(u)$ for all $u$. This means that the Sublocals are a dual order to the nuclei.
\end{definition}


\begin{lemma}[factorisation]
(Leroy Lemme 2)
    \label{lem:factorisation}
    \uses{def:f_star, def:embedding}
    \lean{factorisation}
    \leanok
    Let $i : X \rightarrow E$ be an embedding and $f: Y \rightarrow E$ be a morphism of spaces. To have $f$ factor through $i$, it is necessary and sufficient that $i_*i^*(V) \le f_*f^*(V) for all V \in O(E)$.
\end{lemma}
\begin{proof}
    \leanok
\end{proof}

\subsection{(1.4) Sublocal Union and Intersection}

\begin{definition}[Union of Sublocals]
(Leroy CH 1.4)
    \label{def:sublocal_union}
    \uses{def:sublocal_inclusion,lem:nucleus}
    \leanok
    Let $(X_i)_i$ be a family of sublocals of $E$ and $(e_i)_i$ the corresponding nuclei.
    For all $V \in O(E)$, let $e(V)$ be the union of all $W \in O(E)$ which are contained in all $e_i(V)$.
\end{definition}

\begin{lemma}[Union of Sublocals]
(Leroy CH 4)
    \label{lem:sublocal_union}
    \uses{lem:nucleus, def:sublocal_union}
    \leanok
    \lean{Sublocale.sSup_apply}
    Let $X_i$ be a family of subframes of $E$ and $e_i$ be the corresponding nuclei. For every $V \in O(E)$, let
    $e(V)$ be the union of all $W \in O(E)$ which are contained in every $e_i(V)$. Then \\
    \begin{enumerate}
        \item $e$ is the corresponding nucleus of a sublocale $X$ of $E$
        \item a sublocale $Z$ of $E$ contains $x$ if and only if it contains all $X_i$. $X$ is thus called the union of
        $X_i$ denoted by $\bigcup_i X_i$
    \end{enumerate}
\end{lemma}
\begin{proof}
    The properties of the nucleus (idempotent, increasing, preserving intersection) can be verified by unfolding the
    definition of $e(V)$.
    \leanok
\end{proof}

\begin{lemma}[Sublocal Union equals Nucleus Intersection]
    \label{lem:sublocal_union_nucleus_intersection}
    \uses{lem:sublocal_union,lem:nucleus_intersection}
    \leanok
    For a family of sublocals $X_i$ of $E$, the union $\bigcup X_i$ is the intersection of the corresponding nuclei.
    TODO lean link
\end{lemma}
\begin{proof}
    The infimum of the Nuclei is a Supremum of the sublocals, because the Nuclei are a dual order to the sublocals.)
    This means that it suffices to show that suprema are unique. \\
    TODO Quelle \url{https://proofwiki.org/wiki/Infimum_is_Unique} \\
    Suppose there are two different suprema $c$ and $c'$ of a set $S$. Because of the definition of a supremum, we
    that they are both upper bounds of $S$. But we also know that the supremum is smaller than any other upper bound, so
    we get $c \le c'$ and $c' \le c$. This means that $c = c'$.
    \leanok

\end{proof}


\begin{definition}[Intersection of Sublocals]
    \label{def:sublocal_intersection}
    \uses{lem:sublocal_union,lem:nucleus,lem:nucleus_complete_lattice}
    \lean{Sublocale.inf_apply}
    \leanok
    Let $(X_i)_i$ be a family of sublocal of $E$ and $(e_i)_i$ the corresponding nuclei.
    For all $V \in O(E)$, the intersection $\bigcap X_i$ is the Union of all Nuclei $w$ such that $w \le x_i $ for all $x_i \in X_i $
\end{definition}

\begin{lemma}[Nucleus Complete Lattice]
    \mathlibok
    \label{lem:nucleus_complete_lattice}
    \uses{lem:nucleus_intersection}
    \lean{Nucleus.instCompleteLattice}
    \leanok
    The Nuclei (and therefore the sublocals) form a complete lattice.
\end{lemma}
\begin{proof}
    One can prove that the Nuclei are closed under arbitrary intersections by unfolding the definition of the intersection. The supremum is defined as the infimum of the upper Bound. \\
    \leanok
\end{proof}

\begin{proposition}[Complete Heyting Algebra]
    \label{prop:complete_heyting_algebra}
    \mathlibok
    \lean{Order.Frame}
    \leanok
    A complete Lattice is a Frame if and only if it as a Heyting Algebra.
\end{proposition}
\begin{proof}
(Source Johnstone:) The Heyting implication is right adjoint to the infimum. This means that the infimum preserves
Suprema, since it is a left adjoint.
    \leanok
        \mathlibok

\end{proof}


\begin{lemma}[Nucleus Heyting Algebra]
    \label{lem:nucleus_heyting_algebra}
    \uses{lem:nucleus_intersection}
    \leanok
    \mathlibok
    \lean{Nucleus.instHeytingAlgebra}
    The Nuclei form a Heyting Algebra.
\end{lemma}
\begin{proof}
    \leanok
    Quelle Johnstone
\end{proof}


\begin{lemma}[Nucleus Frame]
    \label{lem:nucleus_frame}
    \mathlibok
    \uses{lem:nucleus_complete_lattice,lem:nucleus_heyting_algebra,prop:complete_heyting_algebra}
    % \lean{Nucleus.instFrame}
    \leanok
    The Nuclei  form a frame.
\end{lemma}
\begin{proof}
    \leanok
\end{proof}

\begin{comment}
\subsection{(1.5) Direct Images}

\begin{definition}[Direct Images]
    \uses{def:sublocal,,lem:factorisation, def:sublocal_intersection}
    \label{def:direct_images}
    Let $f : E \rightarrow F$ be a morphism of Frames. The map $f_*f^* : O(F) \to O(F)$ is the nucleus of the subframe
    $Im(f)$ of $F$. By (lemma 2), $Im(F)$ is the smallest subframe of $F$ through which $f$ can be factored. For any subframe $X$ of $E$, we define the direct image as
    \[f(x) = Im(fi_x)\]
    Where $i_X$ is the inclusion of $X$ into $E$.
\end{definition}

\begin{lemma}[(4) Direct Images Transitive]
(Leroy Lemme 4)
    \label{lem:direct_images_transitive}
    \uses{def:direct_images}
    Given two morphisms $f : E \rightarrow F$ and $g : F \rightarrow G$ and a subspace $X$ of $E$, we have
    \[(gf)(X) = g(f(X))\]
\end{lemma}

\begin{lemma}[(5) Direct Images Families]
(Leroy Lemme 5)
    \label{lem:direct_images_families}
    \uses{def:direct_images, lem:sublocal_union}
    For all morphisms $f: E \rightarrow F$ and a family $(X_i)$ of subspaces of $E$, the following holds:
    \[f(\cup_i X_i) = \cup_i f(X_i)\]
\end{lemma}

\subsection{(6) Inverse Images}
\begin{definition}[Inverse Images]
    \label{def:inverse_images}
    \uses{lem:direct_images_transitive}
    We have a morphism of spaces $f : E \rightarrow F$ and a subspace $Y$ of $F$. The inverse image $f^{-1}(Y)$ is the biggest subspace $X$ of $E$ such that $f(X) \subset Y$. \\
    More generally for a morphism $h : A \rightarrow E$, the necessary and sufficient condition for $h$ to factor through $f^{-1}$ is that $fh$ factors through $Y$.
    \[Im h \subset f^{-1}(Y)\iff f(Im h) \subset Y \iff Im(fh) \subset Y \] \
\end{definition}
\end{comment}
\subsection{(7) Open Sublocals}

\begin{definition}[$e_U$]
    \label{def:e_U}
    \uses{def:nucleus,def:sublocal,lem:sublocal_union, def:sublocal_intersection}
    \lean{Open.toSublocale}
    \leanok
    Let $E$ be a space with $U, H \in O(E)$. We donote by $e_U$ the largest $W \in O(E)$ such that $W \cap U \subset H$. We verify that $e_U$ is the nucleus of a subspace, which we will temporarily denote by $[U]$.
\end{definition}

\begin{lemma}[$e_U$ is a nucleus]
    \label{lem:e_U_nucleus}
    \uses{def:e_U}
    \lean{Open}
    \leanok
    The map $e_U$ is a nucleus.
\end{lemma}
\begin{proof}
    \leanok
\end{proof}


\begin{definition}[Open sublocal]
    \label{def:open_sublocal}
    \uses{def:e_U,lem:e_U_nucleus}
    \lean{Open}
    \leanok
    For any $U \in O(E)$, the sublocal $[U]$ is called an open sublocal of $E$.
   % (+Senf: stimmt das mit dem üblichen überein???)
\end{definition}


\begin{lemma}[(6,7) Open Sublocal Properties]
(Leroy Lemma 6,7)
    \label{lem:sublocal_properties}
    \lean{Open.preserves_inf}
    \leanok
    \uses{def:e_U,def:open_sublocal}
    \begin{enumerate}
        \item For all subspaces $X$ of $E$ and any $U \in O(E)$:
        \[X \subset [U] \iff e_X(U) = 1_E\]

        \item For all $U, V \in O(E)$, we have:
        \[[U \cap V] = [U] \cap [V]\]
        \[e_{U \cap V} = e_Ue_V=e_Ve_U\]
        \[U \subset V \iff [U] \subset [V]\]
        \item
        For all families $V_i$ of elements of $O(E)$, we have:
        \[\cup_i[V_i] = [\cup_iV_i]\]

        \item
 %       For all morphisms of spaces $f: E \to F$ and all $V \in O(E)$, we have:
 %       \[f^{-1}([V])] = [f^*(V)]\]
 %       \item
 %       Let $X$ be a subspace of $E$ and $U \in O(E)$. For all $V \in O(E)$, we have:
 %       \[V \subset e_X(U) \iff [V] \cap X \subset [U]\]
    \end{enumerate}
\end{lemma}
\begin{proof}
    % Fast fertig
    \leanok
\end{proof}


\begin{definition}[Complement]
    \label{def:complement}
    \uses{def:open_sublocal}
    \lean{complement}
    \leanok
    The complement of an open sublocal $U$ of $X$ is the sublocal $X \setminus U$.
    (Leroy p. 12) (+ Senf brauchen wir das allgemein??)
\end{definition}

\begin{lemma}[Complement Injective]
    \label{lem:complement_injective}
    \uses{def:complement}
    \lean{Open.toSublocale_injective}
    \leanok
    The complement is injective.
\end{lemma}
\begin{proof}
    \leanok
\end{proof}

\begin{definition}[Closed Sublocal]
    \label{def:closed_sublocal}
    \uses{def:complement}
    \lean{Closed}
    \leanok
    A sublocal $X$ of $E$ is called closed if $X = E \setminus U$ for some open sublocal $U$ of $E$.
\end{definition}

\begin{lemma}[Intersection of Closed Sublocals]
    \label{lem:closed_intersection}
    \uses{def:closed_sublocal}
    \lean{Closed.instInfSet}
    \leanok
    For any family $X_i$ of closed sublocals of $E$, the intersection $\bigcap X_i$ is closed (it can be computed by
    taking the complment of the union of the complements).
\end{lemma}
\begin{proof}
    \leanok
\end{proof}


\begin{lemma}[(1.8) Properties of Complements]
    \label{lem:properties_of_complements}
    % benutzt vlt complement_injective
    \uses{def:complement}
    \leanok
    For any open sublocal $V$ of $E$ and any sublocal $X$ of $E$, we have:
    \[V \cup X = E \iff E \setminus V subset X\]
    \[V \cap X = \emptyset \iff X \subset E \setminus V\]
    And thereby:
    \[(E - U = E - V) \implies U = V\]
\end{lemma}

\begin{comment}
\begin{lemma}[(1.9) Preimage of complements]
    \label{lem:preimage_of_complements}
    \uses{def:complement,lem:properties_of_complements}
    For any morphism of spaces $g: A \to E$ and any open sublocal $V$ of $F$, we have:
    \[g^{-1}(E - V) = A - g^{-1}(V)\]
\end{lemma}
\end{comment}

\begin{lemma}[(1.8bis) Properties of Complements Part 2]
    \label{lem:properties_of_complements_part_2}
    \uses{def:complement, lem:properties_of_complements}
    \lean{sup_compl_eq_top_iff}
    \leanok
    For any open sublocal $V$ of $E$ and any sublocal $X$ of $E$, we have:
    \[V \cup (E - V) = E \iff V \subset X \]
    \[V \cap (E - V) = \emptyset X \subset V\]
\end{lemma}



\begin{comment}
\begin{lemma}[(1.11) Distribution of Intersections over Unions]
    \label{lem:distribution_of_intersections_over_unions}
    \uses{lem:open_closed_intersection}
    Let $X, Y, L$ be three sub locals of $E$. If $L$ is open or closed, we have:
    \[L \cap (X \cap Y) = (L \cap X) \cup (L \cap Y)\]
\end{lemma}
\end{comment}
\begin{definition}[Further Topology]
    \label{def:further_topology}
    \uses{def:open_sublocal,def:closed_sublocal, def:complement, lem:closed_intersection}
    \lean{Sublocale.closure}
    \leanok
    \begin{enumerate}
        \item $Int X$ is the largest open sublocal contained in $X$
        \item $Ext X$ is the largest open sublocal contained in $E \setminus X$
        \item $\bar{X}$ is the smallest closed sublocal containing $X$
        \item $\partial X = \bar{X} \cap (E - Int X)$
    \end{enumerate}
\end{definition}

\begin{lemma} [Properties of Further Topology]
    \label{lem:properties_of_further_topology}
    \uses{def:further_topology,lem:properties_of_complements_part_2}
    \leanok
    %\lean{closure_eq_compl_ext}
    \begin{enumerate}
        \item $\bar{X} = E \setminus Ext(X)$
        \item $\partial X = E \setminus (Int X \cup Ext X)$
        \item $Int X \cup \partial X = \bar X$
        \item $Ext X \cup \partial X = E \setminus Int X$
    \end{enumerate}
\end{lemma}

\begin{proof}
    \uses{lem:properties_of_complements}
    \leanok
\end{proof}


%\begin{definition}[gamma]
%    \label{def:gamma}
%    $\gamma(E)$ is the minimal element of all dense sublocals of $E$.
%\end{definition}
